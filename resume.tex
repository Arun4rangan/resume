\documentclass{resume}

\usepackage[left=0.5in,top=0.25in,right=0.5in,bottom=0.25in]{geometry}
\usepackage[colorlinks=true, linkcolor=black, urlcolor=black]{hyperref}

\name{Arun Ranganathan}
\address{\href{mailto:arun.ranga@hotmail.ca}{arun.ranga@hotmail.ca}\ \\
         \href{https://www.linkedin.com/in/arun-ranganathan-860674b1/}{in/arunranganathan} \\
         \href{https://github.com/Arun4rangan}{github.com/Arun4rangan}}

\vspace{-1em}
\begin{document}
  \begin{rSection}{Summary}
    \begin{rSummary}
    {
      \item Experience building distributed systems, real-time applications, computer vision applications, \\ android development.
      \item C\texttt{++} (Linux \& Windows), Python, Java (Android), JavaScript, SQL, Golang, C
      \item Node, Django, Angular 2, AngularJS, CloudSQL, PostgreSQL, MySQL, RabbitMQ, Redis, Kubernetes,\\ OpenShift, Celery, gRPC, TensorRT, TensorFlow, OpenVINO, OpenCV, NumPy
    }
    \end{rSummary}
  \end{rSection}
  \begin{rSection}{Education}
    {\bf University of Waterloo} \hfill {\em April 2019} \\ 
    { Bachelor of Applied Science in Honours Nanotechnology Engineering With Distinction }
  \end{rSection}
  \begin{rSection}{Experience}

    \begin{rSubsection}{Paravision}{April 2019 – Present}{Computer Vision Engineer}{Toronto, ON}
    \item Architect parallel system for face detection on video streams. Reduced IPC bottleneck with MMAP and asynchronous programming, amplifying number of frames processed from 40 fps to 100 fps.
    \item Create ETL pipeline on Kubernetes on GCP for image datasets. Quadruple throughput for pre-processing images, processed 8 million images per day.
    \item Configure Openshift platform to host micro-services on bare-metal.
    \item Engineered RESTful features such as grouping and identities on face recognition.
    \item Experience building real-time computer vision systems using OpenCV, Tensorflow, and with accelerators such as TensorRT and OpenVINO on Windows and Linux OS.
    \end{rSubsection}
    \vspace{-1em}
    \begin{rSubsection}{The League}{May 2018 – Augest 2018}{Backend Developer}{San Francisco, CA}
    \item Develop core micro-services that replaced parts of monolithic backend deployed in GCP Kubernetes.
    \item Reduce number of calls to PostgreSQL by implementing trigger based caching.
    \item Create unit and E2E testing to support integration of micro-services.
    \end{rSubsection}
    \vspace{-1em}
    \begin{rSubsection}{The League}{January 2018 – April 2018 }{Android Engineer}{San Francisco, CA}
    \item Rewrite the messaging platform. Optimize how messages are received, stored and managed.
    \item Increase user activity by developing custom photo-messaging capabilities in chats and optimizing the loading and uploading of photo-messages.
    \end{rSubsection}
    \vspace{-1em}
    \begin{rSubsection}{Aspire Financial Technologies}{Augest 2016 – April 2017}{Backend Developer}{Toronto, ON}
    \item Develop query-able RESTful API endpoints on Django.
    \item Engineer aggregating algorithms that compile into complex financial math on PostgreSQL and optimized it to run in real time.
    \end{rSubsection}
  \end{rSection}

  \begin{rSection}{Open Source / Projects}
    \begin{rProjectSection}{\href{https://github.com/redis/redis}{\textbf{Redis}}}{In-memory Database}
      \item Show threading configuration in INFO output \href{https://github.com/redis/redis/pull/7446}{ \textbf{\#7446} } \& \href{https://github.com/redis/redis-doc/pull/1353}{\textbf{\#1353}}
    \end{rProjectSection}
    \vspace{-1em}
    \begin{rProjectSection}{\href{https://github.com/cockroachdb/cockroach}{\textbf {CockroachDB}}}{Distributed SQL Database}
      \item Add support for Pow binary operator for vectorized execution engine \href{https://github.com/cockroachdb/cockroach/pull/50143}{\textbf{\#50143}}
      \item Add ability insert DB2 formatted timestamps \href{https://github.com/cockroachdb/cockroach/pull/50011}{\textbf{\#50011}}
      \item Coerce invalid geography coordinates into geometry \href{https://github.com/cockroachdb/cockroach/pull/50457}{\textbf{\#50457}}
    \end{rProjectSection}
    \vspace{-1em}
    \begin{rProjectSection}{\href{https://github.com/pinojs/pino}{\textbf{Pino}}}{Low overhead Node.js Logger}
      \item Allow custom levels to override default and allow to remove default levels \href{https://github.com/pinojs/pino/pull/515}{ \textbf{\#515} }
      \item Add ability to change level name when creating the instance \href{https://github.com/pinojs/pino/pull/515}{ \textbf{\#503} }
    \end{rProjectSection}
  \end{rSection}

\end{document}
